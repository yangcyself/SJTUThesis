% 载入 SJTUThesis 模版
% \documentclass[degree=doctor, zihao=-4, language=english, review]{sjtuthesis}
\documentclass[degree=master, zihao=-4]{sjtuthesis}
% \documentclass[degree=bachelor, openany, oneside]{sjtuthesis}
% \documentclass[degree=course, language=english, openright, twoside]{sjtuthesis}
% 选项
%   degree=[doctor|master|bachelor|course],   % 可选(默认:doctor),学位类型
%   zihao=[-4|5],                             % 可选(默认:5),正文字号大小
%   language=[chinese|english],               % 可选(默认:chinese),论文的主要语言
%   review,                                   % 可选(默认:关闭),盲审模式

\input{sjtusetup}

\begin{document}

% 无编号内容:中英文论文封面、授权页
\maketitle
\makeorigpage[scans/originality.pdf]
\makeauthpage

% 使用罗马数字对前言编号
\frontmatter

% 摘要
\input{contents/abstract}

% 目录、插图目录、表格目录
\tableofcontents
\listoffigures
\listoftables
\listofalgorithms

% 主要符号、缩略词对照表
\input{contents/nomenclature}

% 使用阿拉伯数字对正文编号
\mainmatter

% 正文内容
\input{contents/intro}
\input{contents/floats}
\input{contents/math_and_citations}
\input{contents/summary}

% 使用英文字母对附录编号
\appendix

% 附录内容,本科学位论文可以用翻译的文献替代。
\input{contents/app_maxwell_equations}
\input{contents/app_flow_chart}

% 文后无编号部分
\backmatter

% 参考资料
\printbibliography[heading=bibintoc]

% 用于盲审的论文需隐去致谢、发表论文、参与项目、申请专利、简历

% 致谢
\input{contents/acknowledgements}

% 发表论文、参与项目、申请专利、简历
% 盲审论文中,发表学术论文及参与科研情况等仅以第几作者注明即可,不要出现作者或他人姓名
\input{contents/publications}
\input{contents/achievements}
\input{contents/resume}

% 中文学士学位论文要求在最后有一个英文大摘要,单独编页码,英文学士学位论文不需要
\input{contents/end_english_abstract}

\end{document}
